\newcommand{\institut}{Institut f\"ur Energie und  Automatisiertungstechnik}
\newcommand{\fachgebiet}{Elektronische Mess- und Diagnosetechnik}
\newcommand{\veranstaltung}{Praktikum Messdatenverarbeitung}
\newcommand{\pdfautor}{\"Ozg\"u Dogan (326 048), Timo Lausen (325 411), Boris Henckell (325 779)}
\newcommand{\autor}{\"Ozg\"u Dogan (326 048)\\ Timo Lausen (325 411)\\ Boris Henckell (325 779)}
\newcommand{\pdftitle}{Praktikum Messdatenverarbeitung  Termin 7}
\newcommand{\prototitle}{Praktikum Messdatenverarbeitung \\ Termin 7}
\newcommand{\aufgabe}{}

\newcommand{\gruppe}{Gruppe: G1 Fr 08-10}
\newcommand{\betreuer}{Betreuer: J\"urgen Funk}

\input{../../packages/tu_header_9}
\begin{document}

% \lstlistoflistings
\definecolor{darkgray}{rgb}{0.95,0.95,0.95}
\definecolor{darkolivegreen}{HTML}{01a801}
\definecolor{functionsBlue}{HTML}{32b9b9}
\definecolor{variableRed}{rgb}{1,0,0}
\definecolor{stringBrown}{HTML}{bc8e8e} % f geht nicht



% \lstset{
%         %\lstset{extendedchars=true} % Umlaute an der richtigen stelle und nicht am Anfang ausgeben
%         %basicstyle=\footnotesize\ttfamily,
%         basicstyle=\small,
%         %
%         inputencoding=utf8,
%         %
%         tabsize=4,
%         showspaces=false,
%         showtabs=false,
%         showstringspaces=true, % no special string spaces
%         %
%         backgroundcolor=\color{darkgray}, % background
%         stringstyle=\color{stringBrown}\fseries, % Strings
%         keywordstyle=\color{functionsBlue}\bfseries, % keywords Blau
%         identifierstyle=\color{variableRed}, % variablen
%         commentstyle=\color{darkolivegreen}, %  comments
%         %
%         breaklines=true,
%         %
%         numbers=left,
%         numberstyle=\tiny,
%         stepnumber=1,
%         numbersep=7pt,
%         %
%         frame=single,
%         columns=flexible,
%         %
%         xleftmargin=-2cm,
%         xrightmargin=-1.5cm,
%         %
%         language=Matlab
% }



%---------------------------------------------------------------------
%---------------------------------------------------------------------
%---------------------------------------------------------------------


\section{Vorbereitungsaufgaben}
\begin{quote}
    
    \subsection{Vorbereitungsaufgaben zu Termin 7}
    \begin{quote}
        
        \subsubsection{Chirp-Signal erzeugen}
        \begin{quote}
        
        Als erstes sollte anhand Matlab ein chirp-Signal erzeugt und untersucht
        werden. Dieser wurde mit dem Aufruf chirp() erzeugt, dem man einen
        Zeitvektor, die Startfrequenz und weitere Angaben über den Verlauf geben
        konnte. Bei unserem Signal sollte ein linearer Frequenzanstieg erfolgen. 
        
        Als Beispiel wurde folgendes Signal erstellt:\\
        t = $0:0.001:2$\\
        chirp(t,$0$,$1$,$100$)\\
        
        Das Chirp-Signal und das Spektrogram dazu sehen so aus:
        
        \begin{center}
                \begin{tabular}{ll}
    
                \hspace{-12em}
                    \begin{minipage}{0.6\textwidth}
    
                        \begin{figure}[H]
                            \label{fig:}
                            \includegraphics[scale=0.63, trim = 3cm 9cm 3cm
                            9cm, clip]{./Bilder/Termin7/bsp_chirp}
                            %FIXME [width=640px,
                             %height=474px]
                            \caption{erzeugtes Chirp-Signal über der Zeit}
                        \end{figure}
    
                    \end{minipage}
                    \begin{minipage}{0.6\textwidth}
    
                        \begin{figure}[H]
                            \label{fig:}
                            \includegraphics[scale=0.63, trim = 3cm 9cm 3cm
                            9cm,
                            clip]{./Bilder/Termin7/bsp_chirp_spectrogram}
                            %FIXME [width=640px,
                             %height=474px]
                            \caption{Spektrogramm des erzeugten Chirp-Signals}
                        \end{figure}
                    \vspace{-1.5em}
    
                    \end{minipage}
    
                \end{tabular}
                \end{center}
                
                \vspace{1.5em}
        
        Man sieht einen Sinusverlauf, dessen Frequenz mit der Zeit immer größer
        wird. Wir vermuten einen linearen Abstieg der Frequenz. Im Spektrogramm
        kann man deutlich sehen, dass nach einer Sekunde die Frequenz den
        erwünschten Wert von $100 Hz$ annimmt.\\
        
        Weiterhin kann man die Auswirkung der Eingabevariablen des
        Spektrogrammaufrufs auf das entstehende Spektrogramm untersuchen. Hier
        sind zwei Beispiele, in denen einmal die Überlappungsfläche zwischen
        zwei Segmenten verkleinert wird und einmal die verwendete Fenstergröße
        vergrößert wird.
        
        
            \begin{center}
                \begin{tabular}{ll}
    
                \hspace{-12em}
                    \begin{minipage}{0.6\textwidth}
    
                        \begin{figure}[H]
                            \label{fig:}
                            \includegraphics[scale=0.63, trim = 3cm 9cm 3cm
                            9cm,
                            clip]{./Bilder/Termin7/bsp_chirp_spectrogram_kleineUeberlappung}
                            %FIXME [width=640px,
                             %height=474px]
                            \caption{Überlappung zwischen den Segmenten im
                            Spektrogram wird kleiner gewählt}
                        \end{figure}
    
                    \end{minipage}
                    \begin{minipage}{0.6\textwidth}
    
                        \begin{figure}[H]
                            \label{fig:}
                            \includegraphics[scale=0.63, trim = 3cm 9cm 3cm
                            9cm,
                            clip]{./Bilder/Termin7/bsp_chirp_spectrogram_grossesFenster}
                            %FIXME [width=640px,
                             %height=474px]
                            \caption{verwendete Fensterfolge wird größer
                            gewählt}
                        \end{figure}
                    \vspace{-1.5em}
    
                    \end{minipage}
    
                \end{tabular}
                \end{center}
                
                \vspace{1.5em}
        
        Innerhalb des gewählten Fensters, kann das verwendete Signal als
        stationär angenommen werden. Wird das Beobachtungsfenster kürzer, nimmt
        auch die Frequenzauflösung ab. Wird das Fenster zu groß, kann wiederum
        das Signal innerhalb des Fensters nicht mehr als stationär angenommen
        werden. Die Verschlechterung und Ungenauigkeiten im Spektrogramm kann
        man in den Plots oben gut erkennen.
        
        \end{quote}%ende Chirp-Signal erzeugen
        
        \subsubsection{Matlab-Funktion: Frequenzverlau über der Zeit}
        \begin{quote}
        
        Als nächstes sollte eine Matlab-Funktion erstellt werden, die im
        Zeitbereichdie momentane Frequenz ermittelt. Dieser berechnete
        Frequenzverlauf über der Zeit sollte geplottet und mit dem erwarteten
        Verlauf verglichen werden. In die erstellte Funktion kann in den Codes am
        Ende des Protokolls eingesehen werden. Der entstandene Plot ist hier zu
        sehen:
        
        \begin{figure}[H]
                    \centering
                        \includegraphics[scale=0.7, trim = 1cm 9cm 1.5cm 8cm,
                        clip]{./Bilder/Termin7/chirpsignal_vs_frequenzanstieg}
                            \caption{Chirp-Signal und der dazugehörige
                            Frequenzanstieg, berechnet aus dem Zeitsignal}
                            \label{fig:./Bilder/chirpsignal_vs_frequenzanstieg}
                    \end{figure}
        \end{quote}
        
        Da beim Erstellen des Chirpsignals ein linearer Ansteig der Frequenz
        vorausgesetzt war, wurde auch ein linearer Frequenzverlauf erwartet. Das
        Ergebnis bestätigt die Erwartung.
        
        \end{quote}% ende Matlab-Funktion: Frequenzverlauf über der Zeit
        
        \subsubsection{Matlab-Funktion: Frequenzverlauf anhand Spektrogram}
        \begin{quote}
        
        Nun sollte der gleiche Frequenzverlauf anhand des Spektrogramms ermittelt
        werden. Dafür implementierten wir eine weitere Matlab-Funktion, welche in
        den Codes zu sehen ist. Das erwartete Ergebnis war wieder ein positiver
        linearer Anstieg.
        
        \begin{figure}[H]
                    \centering
                        \includegraphics[scale=0.7, trim = 1cm 9cm 1.5cm 8cm,
                        clip]{./Bilder/Termin7/chirpsignal_vs_frequenzanstieg_spectrogram}
                        \caption{Chirp-Signal und der dazugehörige
                            Frequenzanstieg, berechnet aus dem Spektrogramm}
                            \label{fig:./Bilder/chirpsignal_vs_frequenzanstieg_spectrogram}
                    \end{figure} 
            
        \end{quote}% ende Matlab-Funktion: Frequenzverlauf anhand Spektrogram
         
        \subsubsection{Drehzahl-Berechnung anhand Amplitudenspektrum des
        Motorstroms}
        \begin{quote}
        
        Zuletzt wurde die Drehzahl des verwendeten Motors berechnet. Dafür wurden
        uns drei Messreihen mit jeweils Strom- und Tachomesswerten vorgegeben.
        Zunächst sollten die Amplitudenspektren der Motorströme erstellt werden,
        welche uns durch die DFT der Stromwerte gelang. Die erhöhten Amplitudenwerte entsprachen
        dabei den einfachen Vielfachen der Drehfrequenz des Motors mit der
        jeweiligen Versorgungsspannung. Die deutlich herausstehenden Peaks dagegen
        traten nur bei dem 18- oder 36-fachen Vielfachen der Drehfrequenz auf.
        Dieses Wissen nutzen wir, indem wir anhand eines Matlabalgorithmuses den
        höhsten Peak mit Index ausgaben lassen (die höchsten Peaks waren stets
        jene, die am nächsten zu der null standen), um daraus eine Frequenz zu
        bestimmen und durch 18 zu teilen. So berechneten wir die drei geforderten
        Drehfrequenzen. Der Algorithmus steht in den Codes, die drei
        Amplitudenspektren sehen folgendermaßen aus:
        
         \begin{center}
                \begin{tabular}{ll}
    
                \hspace{-12em}
                    \begin{minipage}{0.6\textwidth}
    
                        \begin{figure}[H]
                            \label{fig:}
                            \includegraphics[scale=0.63, trim = 3cm 9cm 3cm
                            8.5cm, clip]{./Bilder/Termin7/ampl_spektrum_messung1}
                            %FIXME [width=640px,
                             %height=474px]
                            \caption{Spektrum des Motorstroms bei 10V}
                        \end{figure}
    
                    \end{minipage}
                    \begin{minipage}{0.6\textwidth}
    
                        \begin{figure}[H]
                            \label{fig:}
                            \includegraphics[scale=0.63, trim = 3cm 9cm 3cm
                            8.5cm,
                            clip]{./Bilder/Termin7/ampl_spektrum_messung2}
                            %FIXME [width=640px,
                             %height=474px]
                            \caption{Spektrum des Motorstroms bei 20V}
                        \end{figure}
                    \vspace{-1.5em}
    
                    \end{minipage}
    
                \end{tabular}
                \end{center}
               
               
               \begin{figure}[H]
                    \centering
                        \includegraphics[scale=0.63, trim = 1cm 9cm 1.5cm 8cm,
                        clip]{./Bilder/Termin7/ampl_spektrum_messung3}
                        \caption{Spektrum des Motorstroms bei 30V}
               \end{figure}
        
        Die berechneten Drehfrequenzen betragen für die erste Messung ($10V$) =
        $19.5627 Hz$, für die zweite Messung ($20V$) = $56.8131 Hz$ und für die
        dritte Messung ($30V$) = $94.31343 Hz$.\\
        
        Außerdem wurden die Drehfrequenzen auch aus den jeweiligen Tachosignalen
        bestimmt. Dafür wurde genauso vogegangen wie bei den Motorströmen. Die
        Spektren sind hier:
           
        \begin{center}
                \begin{tabular}{ll}
    
                \hspace{-12em}
                    \begin{minipage}{0.6\textwidth}
    
                        \begin{figure}[H]
                            \label{fig:}
                            \includegraphics[scale=0.45, trim = 0.8cm 7cm 3cm
                            8.5cm, clip]{./Bilder/Termin7/ampl_spektrum_messung1_tacho}
                            %FIXME [width=640px,
                             %height=474px]
                            \caption{Spektrum des Tachosignals bei 10V}
                        \end{figure}
    
                    \end{minipage}
                    \begin{minipage}{0.6\textwidth}
    
                        \begin{figure}[H]
                            \label{fig:}
                            \includegraphics[scale=0.45, trim = 0.8cm 7cm 3cm
                            8.5cm,
                            clip]{./Bilder/Termin7/ampl_spektrum_messung2_tacho}
                            %FIXME [width=640px,
                             %height=474px]
                            \caption{Spektrum des Tachosignals bei 20V}
                        \end{figure}
                    \vspace{-1.5em}
    
                    \end{minipage}
    
                \end{tabular}
                \end{center}
               
               
               \begin{figure}[H]
                    \centering
                        \includegraphics[scale=0.45, trim = 0.8cm 7cm 1.5cm 8cm,
                        clip]{./Bilder/Termin7/ampl_spektrum_messung3_tacho}
                        \caption{Spektrum des Tachosignals bei 30V}
               \end{figure}
               
        Die anhand der Tachosignale berechneten Drehfrequenzen betragen für die
        erste Messung ($10V$) = $19.5627 Hz$, für die zweite Messung ($20V$) =
        $56.8131 Hz$ und für die dritte Messung ($30V$) = $94.3134 Hz$.\\
        
        
        \end{quote}%ende Drehzahl-Berechnung anhand Amplitudenspektrum des
        % Motorstroms
        
        
    \end{quote}
    
    \subsection{Vorbereitungsaufgaben zu Termin 8}
    \begin{quote}
    
        \subsubsection{Zerlegung des Signals mittels Haar-Tranformation}
        \begin{quote}
        
        In der ersten Vorbereitungsaufgabe des 8.Praktikumstermins wird eine
        vorgegebene strom.m Datei verwendet, welche einen angechnittenen Sinus
        im Bereich $[0,8\pi]$ darstellt. Dieses Signal wird mit der Schnellen
        Haar-Transformation zerlegt. Außerdem wird die Approximation und die
        Details dür die Skalierungen $m = 1 \ldots 5$ berechnet werden, wofür
        die drei Funktionen haardec.m, haardeclevel.m und getAppDet.m
        implementiert werden. Diese stehen unter den Codes.
        
        \end{quote}%ende Zerlegung des Signals mittels Haar-Transformation
        
        
        \subsubsection{Darstellung der Approximationen}
        \begin{quote}
        
        Als nächstes werden die Approximationen und die Details des
        angeschnittenen Sinus-Signals dargestellt und das stationäre Spektrum
        sowie das Spektrogramm berechnet. Diese spektralen Darstellungen werden
        mit den Darstellungen mittels Wavelets verglichen. Am Ende soll
        ermittelt werden, welche Darstellung bestimmte Informationen über das
        verwendete Signal besser veranschaulicht.
        
        
        \begin{figure}[H]
                    \centering
                        \includegraphics[scale=0.4, trim = 1cm 7cm 1.5cm 8cm,
                        clip]{./Bilder/Termin8/Spektrum}
                        \caption{Spektrum des angeschnittenen Sinus-Signal}
                    \end{figure} 
        
        
        Im stationären Spektrum des gegebenen Signals kann man die
        Frequenzanteile mit ihren entsprechenden Amplituden erkennen.
        
        \begin{figure}[H]
                    \centering
                        \includegraphics[scale=0.4, trim = 1cm 7cm 1.5cm 8cm,
                        clip]{./Bilder/Termin8/Spectrogam}
                        \caption{Spektrogram des angeschnittenen Sinus-Signal}
                    \end{figure} 
        \end{quote}%ende Darstellung der Approximationen
        


        Außerdem kann man im Spektrogram des Signals die durch die Anschneidung
        entstandenen Flanken erkennen. Dies macht sich in dem zeitlich periodischen
        Farbverlauf des Spektrograms deutlich sichtbar.\\
        
        Die Darstellungen der Approximationen und der Details für fünf
        Skalierungslevel sind dabei in der vierten Aufgabe der Vorbereitung
        dargestellt, wo sie auch zeitgleich mit den Zerlegungen des
        Daubechies-Wavelets (db3) verglichen werden. Da eine Zerlegung mittels
        Wavelets eine Tief- und Hochpassfilterung darstellt, kann man es so
        interpretieren, dass die gefilterten tiefen Frequenzen als
        Approximationen und die gefilterten hohen Frequenzen als Details
        abgespeichert werden. Je öfter eine Zerlegung stattfindet, desto öfter
        wird die Approximation hoch- und tiefpassgefiltert, wodurch am Ende in
        dem Approximationsvektor nur noch die aller tiefsten Frequenzen
        enthalten sind.


        \end{quote}%ende Darstellung der Approximationen
        
        \subsubsection{Daubechies-Wavelets}
        \begin{quote}
        
        Nun wird das angeschnitte Sinus-Signal mit Hilfe von Daubechies-Wavelets
        zerlegt. Dazu verwenden wir die Matlab-Fuktionen wavedec, appcoef
        und detcoef und variieren die Anzahl der verschwindenen
        Momente der Wavelets, indem wir bei der wavedec.m Ausführung die
        Wavelets db1, db5, db10 und db15 wählen. Damit werden die
        verschwindenden Momente in jedem Wavelet $1$, $5$, $10$ und $15$
        betragen. Es werden nur die Zerlegungen in Level 2 angezeigt.
        
         
        \begin{center}
                \begin{tabular}{ll}
    
                \hspace{-12em}
                    \begin{minipage}{0.6\textwidth}
    
                        \begin{figure}[H]
                            \label{fig:}
                            \includegraphics[scale=0.45, trim = 0.8cm 6cm 3cm
                            7.5cm,
                            clip]{./Bilder/Termin8/Daubechies_Wavlet_1db_lvl_2}
                            %FIXME [width=640px,
                             %height=474px]
                            \caption{Daubechies-Wavelet mit 1 vanishing moment}
                        \end{figure}
    
                    \end{minipage}
                    \begin{minipage}{0.6\textwidth}
    
                        \begin{figure}[H]
                            \label{fig:}
                            \includegraphics[scale=0.45, trim = 0.8cm 6cm 3cm
                            7.5cm,
                            clip]{./Bilder/Termin8/Daubechies_Wavlet_5db_lvl_2}
                            %FIXME [width=640px,
                             %height=474px]
                            \caption{Daubechies-Wavelet mit 5 vanishing moment}
                        \end{figure}
                    \vspace{-1.5em}
    
                    \end{minipage}
    
                \end{tabular}
                \end{center}
                
                %ende Daubechies-Wavelets mit db1 und db5
                
        \begin{center}
                \begin{tabular}{ll}
    
                \hspace{-12em}
                    \begin{minipage}{0.6\textwidth}
    
                        \begin{figure}[H]
                            \label{fig:}
                            \includegraphics[scale=0.45, trim = 0.8cm 6cm 3cm
                            7.5cm,
                            clip]{./Bilder/Termin8/Daubechies_Wavlet_10db_lvl_2}
                            %FIXME [width=640px,
                             %height=474px]
                            \caption{Daubechies-Wavelet mit 10 vanishing moment}
                        \end{figure}
    
                    \end{minipage}
                    \begin{minipage}{0.6\textwidth}
    
                        \begin{figure}[H]
                            \label{fig:}
                            \includegraphics[scale=0.45, trim = 0.8cm 6cm 3cm
                            7.5cm,
                            clip]{./Bilder/Termin8/Daubechies_Wavlet_15db_lvl_2}
                            %FIXME [width=640px,
                             %height=474px]
                            \caption{Daubechies-Wavelet mit 15 vanishing moment}
                        \end{figure}
                    \vspace{-1.5em}
    
                    \end{minipage}
    
                \end{tabular}
                \end{center}
                
                %ende Daubechies-Wavelets mit db10 und db15
                \vspace{2em}
                
            Man kann sehen, dass sich die Zerlegungen deutlich voneinander
            unterscheiden. Gehen wir zunächst auf die Approximationen ein. Es
            ist deutlich, dass es bei der ersten Darstellung mit nur einem
            verschwindenden Moment kaum einen Unterschied zu einer
            Haar-Wavelet Zerlegung gibt, wohingegen es bei der Zerlegung mit
            fünf verschwindenen Momenten der Verlauf der Approximationen sich
            ein wenig verändert. Wird die Anzahl der verschwindenen Momente auf
            zehn verdoppelt, so wird der Verlauf noch unkontinuierlicher und
            einzelne Amplituden fangen an zu schwanken. Bei fünfzehn bestätigt
            sich diese Beobachtung sogar verstärkter.\\
            Sieht man sich die Details der Zerlegungen an, entdeckt man
            ebenfalls eine Veränderung. Ähnlich wie bei den Approximationen wird
            der kontinuierliche Verlauf mit steigender Anzahl an verschwindenen
            Momenten gestört und die Amplituden variieren und verkleinern sich.
                
                
        \end{quote}%ende Daubechies-Wavelets
        
        
        \subsubsection{Vergleich der Zerlegungen mittels
        Daubechies-Wavelets und Haar-Wavelets}
        \begin{quote}
        
        Zuletzt wird in der Vorbereitung des Praktikums die Zerlegungen mittels
        Daubechies-Wavelets mit den Zerlegungen mittels Haar-Wavelets
        verglichen. Die Unterschiede sind folgende:
        
        \begin{center}
                \begin{tabular}{ll}
    
                \hspace{-8em}
                    \begin{minipage}{0.6\textwidth}
    
                        \begin{figure}[H]
                            \label{fig:}
                            \includegraphics[scale=0.4, trim = 2cm 6cm 1cm
                            7.5cm,
                            clip]{./Bilder/Termin8/Haar_Wavlet_lvl_1}
                            %FIXME [width=640px,
                             %height=474px]
                            \caption{Haar-Wavelet Zerlegung, Level 1}
                        \end{figure}
    
                    \end{minipage}
                    \begin{minipage}{0.6\textwidth}
    
                        \begin{figure}[H]
                            \label{fig:}
                            \includegraphics[scale=0.4, trim = 2cm 6cm 1cm
                            7.5cm,
                            clip]{./Bilder/Termin8/Daubechies_Wavlet_lvl_1}
                            %FIXME [width=640px,
                             %height=474px]
                            \caption{Daubechies-Wavelet Zerlegung, Level 1}
                        \end{figure}
                    \vspace{-1.5em}
    
                    \end{minipage}
    
                \end{tabular}
                \end{center}
                
                %Ende Vergleich Level 1
       
        \begin{center}
                \begin{tabular}{ll}
    
                \hspace{-8em}
                    \begin{minipage}{0.6\textwidth}
    
                        \begin{figure}[H]
                            \label{fig:}
                            \includegraphics[scale=0.4, trim = 2cm 6cm 1cm
                            7.5cm,
                            clip]{./Bilder/Termin8/Haar_Wavlet_lvl_2}
                            %FIXME [width=640px,
                             %height=474px]
                            \caption{Haar-Wavelet Zerlegung, Level 2}
                        \end{figure}
    
                    \end{minipage}
                    \begin{minipage}{0.6\textwidth}
    
                        \begin{figure}[H]
                            \label{fig:}
                            \includegraphics[scale=0.4, trim = 2cm 6cm 1cm
                            7.5cm,
                            clip]{./Bilder/Termin8/Daubechies_Wavlet_lvl_2}
                            %FIXME [width=640px,
                             %height=474px]
                            \caption{Daubechies-Wavelet Zerlegung, Level 2}
                        \end{figure}
                    \vspace{-1.5em}
    
                    \end{minipage}
    
                \end{tabular}
                \end{center}
                
                %Ende Vergleich Level 2
                
                
          \begin{center}
                \begin{tabular}{ll}
    
                \hspace{-8em}
                    \begin{minipage}{0.6\textwidth}
    
                        \begin{figure}[H]
                            \label{fig:}
                            \includegraphics[scale=0.4, trim = 2cm 6cm 1cm
                            7.5cm,
                            clip]{./Bilder/Termin8/Haar_Wavlet_lvl_3}
                            %FIXME [width=640px,
                             %height=474px]
                            \caption{Haar-Wavelet Zerlegung, Level 3}
                        \end{figure}
    
                    \end{minipage}
                    \begin{minipage}{0.6\textwidth}
    
                        \begin{figure}[H]
                            \label{fig:}
                            \includegraphics[scale=0.4, trim = 2cm 6cm 1cm
                            7.5cm,
                            clip]{./Bilder/Termin8/Daubechies_Wavlet_lvl_3}
                            %FIXME [width=640px,
                             %height=474px]
                            \caption{Daubechies-Wavelet Zerlegung, Level 3}
                        \end{figure}
                    \vspace{-1.5em}
    
                    \end{minipage}
    
                \end{tabular}
                \end{center}
    
                %ende Vergleich Level 3
                
                
        \begin{center}
                \begin{tabular}{ll}
    
                \hspace{-8em}
                    \begin{minipage}{0.6\textwidth}
    
                        \begin{figure}[H]
                            \label{fig:}
                            \includegraphics[scale=0.4, trim = 2cm 6cm 1cm
                            7.5cm,
                            clip]{./Bilder/Termin8/Haar_Wavlet_lvl_4}
                            %FIXME [width=640px,
                             %height=474px]
                            \caption{Haar-Wavelet Zerlegung, Level 4}
                        \end{figure}
    
                    \end{minipage}
                    \begin{minipage}{0.6\textwidth}
    
                        \begin{figure}[H]
                            \label{fig:}
                            \includegraphics[scale=0.4, trim = 2cm 6cm 1cm
                            7.5cm,
                            clip]{./Bilder/Termin8/Daubechies_Wavlet_lvl_4}
                            %FIXME [width=640px,
                             %height=474px]
                            \caption{Daubechies-Wavelet Zerlegung, Level 4}
                        \end{figure}
                    \vspace{-1.5em}
    
                    \end{minipage}
    
                \end{tabular}
                \end{center}
    
                %ende Vergleich Level 4
                
                
                    \begin{center}
                \begin{tabular}{ll}
    
                \hspace{-8em}
                    \begin{minipage}{0.6\textwidth}
    
                        \begin{figure}[H]
                            \label{fig:}
                            \includegraphics[scale=0.4, trim = 2cm 6cm 1cm
                            7.5cm,
                            clip]{./Bilder/Termin8/Haar_Wavlet_lvl_5}
                            %FIXME [width=640px,
                             %height=474px]
                            \caption{Haar-Wavelet Zerlegung, Level 5}
                        \end{figure}
    
                    \end{minipage}
                    \begin{minipage}{0.6\textwidth}
    
                        \begin{figure}[H]
                            \label{fig:}
                            \includegraphics[scale=0.4, trim = 2cm 6cm 1cm
                            7.5cm,
                            clip]{./Bilder/Termin8/Daubechies_Wavlet_lvl_5}
                            %FIXME [width=640px,
                             %height=474px]
                            \caption{Daubechies-Wavelet Zerlegung, Level 5}
                        \end{figure}
                    \vspace{-1.5em}
    
                    \end{minipage}
    
                \end{tabular}
                \end{center}
                
                %ende Vergleich Level 5
                
                \vspace{2em}
                
            Zuerst fällt auf, dass die Zerlegung mit unterschiedlichen Wavelets unterschiedliche
            Approximation und daraus resultierend unterschiedliche Details zur
            Folge hat. Des weiteren fällt auf, dass bei der Haar-Wavelet Transformation in der Stufe 5 
            alle 4 Werte konstant 0 sind, während die Werte der Approximation
            bei der Daubechies-Wavelets Transformation unterschiedlich sind.\\
            Da sich jede Approximationsstufe auch als ein Tiefpass interpretiert lässt finden sich in jeder Wavelet
            Stufe Informatoinen über einen speziellen Frequenzbereich. Zum Beispiel zeigen die Details des oben
            gezeigten Stroms klar die Anschnittsmomente des Stroms. 

        \end{quote}%ende Vergleich der Zerlegungen mittels
        %Daubechies-Wavelets und Haar-Wavelets
    
    \end{quote}%Ende Vorbereitungsaufgaben zu Termin 8

\end{quote}%Ende Vorbereitungsaufgaben

\section{Durchführungen}
\begin{quote}

        \subsection{Durchführung zu Termin 7}
        \begin{quote}
            
            
        Bei diesem Versuch soll die Drehzahl eines Universalmotors, der mit Gleichstrom betrieben wird, 
        bestimmt werden. Dazu wird der Strom gemessen und anhand von Oberwellen die Drehzahl bestimmt. 
        Zusätzlich befindet sich ein Tacho an dem Motor, mit dem für Vergleichszwecke 
        dierkt die Drehzahl gemessen wird.
    
         \begin{figure}[H]
                    \centering
                        \includegraphics[scale=0.5, trim = 0cm 0cm 0cm 0cm,
                        clip]{./timo/Versuchasaufbau.png}
                        \caption{Versuchsaufbau der Drehzahlmessung}
                    \end{figure} 
    
        
        Die Spannung, des Tachosignals ist zu groß, um sie direkt auf eine Messkarte zu geben. 
        Daher wird sie vorher mit einem Spannungsteiler im Verhältnis $1:10$
        herruntergeteilt. Um Allaising zu verhindern wird das Tachosignal anschließend mit 
        einem aktiven Filter gefiltert. Dann wird die Spannung mit der Messkarte
        erfasst.\\
        
        Der Strom wird mit einer Stromzange gemessen. Der Stromwandler gibt allerdings wieder 
        nur einen Strom aus. Dieser Strom wird mit einem Widerstand in eine Spannung umgewandelt. 
        Diese Spannung wird ebenfalls gefiltert und mit der 
        Messkarte aufgenommen. Der Filter wird einerseits genutzt, um den Gleichspannungsanteil 
        auszusperren und das Stromsignal zu verstärken, da sonst zu großes 
        Quantisierungsrauschen entsteht.\\

        Bevor der Versuch beginnt, sind die Abtastrate, die Messdauer und die
        Grenzfrequenzen der Antiallaisingfilter zu bestimmen. Die Messkarte besitzt eine maximale 
        Gesamtsamplerate von $200\frac{kS}{s}$. Diese Abtastrate kann auf die einzelnen
        Kanäle verteilt werden.\\
    
        Um eine möglichst hohe Auflösung zu erhalten, wird die maximale Abtastrate genutzt und auf 
        beide Kanäle gleichmäßig aufgeteilt. Jeder Kanal erhält also eine Samplerate
        von $100\frac{kS}{s}$. Um das Abtasttheorem einzuhalten, müssen alle Frequenzen
        über $50kHz$ zu null gedämpft werden. Daher wird die Grenzfrequenz der
        Antiallaisingfilter auf $40kHz$ eingestellt. Dieser wert wurde experimentell
        bestimmt. Für die Messdauer wurde nach einem Probelauf 15 Sekunden.
        gewählt.  
        
            
        \end{quote}%Ende der Durchführung von Termin 7

        \subsection{Durchführung zu Termin 8}
        \begin{quote}
        
            Der Versuchsaufbau im 8.Termin des Praktikums ähnelte sehr dem Aufbau aus 7.Termin. Anstatt den
            Motorstrom und das Tachosignal aufzunehmen, haben wir diesesmal nur den Strom
            (AC- und DC-Anteil) des fehlerfreien Universalmotors aufgenommen. Dieses
            Signal wurde dann mithilfe von Haar- und Daubechies-Wavelets zerlegt
            und untersucht. Dabei sollten geeignete Approximations- und
            Detailsdarstellungen gewählt werden, um den Einschaltzeitpunkt und
            den Drehzahlanstiegt über der Zeit darzustellen.\\
            
            Außerdem wurde der Strom (nur AC-Anteil) des fehlerfreien Motors
            und des Motors mit Lamellenfehler im stationären Zustand
            aufgenommen. Diese Stromsignale der zwei verschiedenen Motoren
            wurden dann anhand der Wavelets auf Unterschiede untersucht.
            Die Ergebnisse werden in der Auswertung diskutiert. 
            
        \end{quote}%Ende der Durchführungen von Termin 8

\end{quote}%Ende Durchführungen


%--------------------------------------------------------------------
%--------------------------------------------------------------------
\section{Auswertung}
\begin{quote}
    \subsection{Auswertung Termin 7}
    \begin{quote}
        
        Zunächst soll die Drehzahl des Motors durch das Tachosignal bestimmt werden. 
        Dazu werden die Funktionen aus den 
        Vorbereitungsaufgaben verwendet. Die Drehzahl wird einmal direkt im Zeitbereich 
        bestimmt und einmal durch eine Auswertung im Frequenzbereich.
    
        \vspace{2em}
        
        \begin{figure}[htb]
        \centering
        \includegraphics[width=1\textwidth]{./timo/zeitsignale.png}
        \caption{Drehzahl im Zeitbereich bestimmt}
        \end{figure}
        
        \vspace{2em}
        
        Die zeitliche Auflösung ist im Zeitbereich sehr gut. Da das Tachosignal nur eine Frequenz 
        zur Zeit enthält, kann diese auch noch gut bestimmt werden. Allerdings ist der
        Verlauf der Drehzahlkurve sprungartig. Die Frequenzauflösung ist also etwas grob. 
        Das Stromsignal enthält viele verschiedene Oberwellen. Die zeitliche Auflösung ist zwar immernoch 
        sehr gut, allerdings können die vielen verschiedenen Frequenzen nicht mehr richtig voneinander 
        getrennt werden. Pro Zeitpunkt ist nur eine Frequenz aus dem Plot ablesbar.
        Dennoch haben beide Kurven die selbe Tendenz. Das Tachosignal, liefert aber 
        wesendlich bessere Ergebnisse.
    
        \begin{figure}[htb]
        \centering
        \includegraphics[width=1\textwidth]{./timo/frequenzsignale.png}
        \caption{Drehzahl im Zeitbereich bestimmt}
        \end{figure}
        
        
        Im Gegensatz zum Zeitbereich, sind im Frequenzbereich mehrere Frequenzen gleichzeitig zu erkennen. 
        Diese sind zeitlich aber nicht mehr so gut aufgelöst. Nun sind im Stromsignal meherer 
        Oberwellen zu erkennen. Die Verläufe der Oberwellen sind gut zu erkennen. Daher kann nun 
        einfach die Drehzahl bestimmt werden, wenn auch nur für 
        Zeitbereich und nicht für Zeitpunkte. Im Gegensatz dazu ist das Tachosignal wesendlich 
        schlechter zu erkennen. Es wurde zeitliche Auflösung aufgegeben um eine bessere Frequenzauflösung 
        zu erhalten. Diese ist aber wertlos, da eh nur eine Frequenz im Signal enthalten ist. 
        Zusätzlich kommt es durch die nicht unendliche Frequenzauflösung dazu, dass 
        die genaue Frequenz nicht genau erkannt werden kann. Das Spektrogramm des
        Tachosignals ist ungeeignet um die Drehzahl des Motors zu messen. Mit dem Spektrogramm des 
        Stroms sind hier wesendlich bessere Ergebnisse zu erzielen.
    
    
        Die zeitliche Messung des Tachosignals und die Frequenzmessung das Stroms ergeben eine 
        Drehzahl von ca. $9600\frac{U}{min}$. Die selbe Tendenz ist mit der
        zeitlichen Messung des Stroms zu erkennen, kann allerdings nicht so genau bestimmt werden. 
        Die Frequenzmessung des Tachosignals ist viel zu ungenau. Es wurde eine ungefähre Drehzahl von 
        $3750\frac{U}{min}$ erkannt, diese weicht aber zu stark von den anderen
        Messungen ab und sollte kritisch betrachtet werden.
    
    
    \end{quote}  % Ende Subsection Auswertung Termin 7

    \subsection{Auswertung Termin 8}
    \begin{quote}
        
        Zunächst werten wir die Messung des Motorstroms am hochfahrenden
        Universalmotor, indem wir das Originalsignal mit einem Haar- und einem
        Daubechies-Wavelet zerlegen und uns die Details und die Approximationen
        bis zum 12. Level angucken. Die geplotteten Ergebnisse sind hier:
        
         \begin{center}
                \begin{tabular}{ll}
    
                \hspace{-8em}
                    \begin{minipage}{0.6\textwidth}
    
                        \begin{figure}[H]
                            \label{fig:}
                            \includegraphics[scale=0.4, trim = 2cm 6cm 1cm
                            7.5cm,
                            clip]{./Bilder/Termin8/fehlerfrei_hochlaufen_Haar_Wavlet_lvl_1}
                            %FIXME [width=640px,
                             %height=474px]
                            \caption{Haar-Wavelet Zerlegung, Level 1}
                        \end{figure}
    
                    \end{minipage}
                    \begin{minipage}{0.6\textwidth}
    
                        \begin{figure}[H]
                            \label{fig:}
                            \includegraphics[scale=0.4, trim = 2cm 6cm 1cm
                            7.5cm,
                            clip]{./Bilder/Termin8/fehlerfrei_hochlaufen_Daubechies_Wavlet_lvl_1}
                            %FIXME [width=640px,
                             %height=474px]
                            \caption{Daubechies-Wavelet Zerlegung, Level 1}
                        \end{figure}
                    \vspace{-1.5em}
    
                    \end{minipage}
    
                \end{tabular}
                \end{center}
                
                %Ende Vergleich Level 1
       
        \begin{center}
                \begin{tabular}{ll}
    
                \hspace{-8em}
                    \begin{minipage}{0.6\textwidth}
    
                        \begin{figure}[H]
                            \label{fig:}
                            \includegraphics[scale=0.4, trim = 2cm 6cm 1cm
                            7.5cm,
                            clip]{./Bilder/Termin8/fehlerfrei_hochlaufen_Haar_Wavlet_lvl_3}
                            %FIXME [width=640px,
                             %height=474px]
                            \caption{Haar-Wavelet Zerlegung, Level 3}
                        \end{figure}
    
                    \end{minipage}
                    \begin{minipage}{0.6\textwidth}
    
                        \begin{figure}[H]
                            \label{fig:}
                            \includegraphics[scale=0.4, trim = 2cm 6cm 1cm
                            7.5cm,
                            clip]{./Bilder/Termin8/fehlerfrei_hochlaufen_Daubechies_Wavlet_lvl_3}
                            %FIXME [width=640px,
                             %height=474px]
                            \caption{Daubechies-Wavelet Zerlegung, Level 3}
                        \end{figure}
                    \vspace{-1.5em}
    
                    \end{minipage}
    
                \end{tabular}
                \end{center}
                
                %Ende Vergleich Level 2
                
                
          \begin{center}
                \begin{tabular}{ll}
    
                \hspace{-8em}
                    \begin{minipage}{0.6\textwidth}
    
                        \begin{figure}[H]
                            \label{fig:}
                            \includegraphics[scale=0.4, trim = 2cm 6cm 1cm
                            7.5cm,
                            clip]{./Bilder/Termin8/fehlerfrei_hochlaufen_Haar_Wavlet_lvl_6}
                            %FIXME [width=640px,
                             %height=474px]
                            \caption{Haar-Wavelet Zerlegung, Level 6}
                        \end{figure}
    
                    \end{minipage}
                    \begin{minipage}{0.6\textwidth}
    
                        \begin{figure}[H]
                            \label{fig:}
                            \includegraphics[scale=0.4, trim = 2cm 6cm 1cm
                            7.5cm,
                            clip]{./Bilder/Termin8/fehlerfrei_hochlaufen_Daubechies_Wavlet_lvl_6}
                            %FIXME [width=640px,
                             %height=474px]
                            \caption{Daubechies-Wavelet Zerlegung, Level 6}
                        \end{figure}
                    \vspace{-1.5em}
    
                    \end{minipage}
    
                \end{tabular}
                \end{center}
    
                %ende Vergleich Level 3
                
                
        \begin{center}
                \begin{tabular}{ll}
    
                \hspace{-8em}
                    \begin{minipage}{0.6\textwidth}
    
                        \begin{figure}[H]
                            \label{fig:}
                            \includegraphics[scale=0.4, trim = 2cm 6cm 1cm
                            7.5cm,
                            clip]{./Bilder/Termin8/fehlerfrei_hochlaufen_Haar_Wavlet_lvl_9}
                            %FIXME [width=640px,
                             %height=474px]
                            \caption{Haar-Wavelet Zerlegung, Level 9}
                        \end{figure}
    
                    \end{minipage}
                    \begin{minipage}{0.6\textwidth}
    
                        \begin{figure}[H]
                            \label{fig:}
                            \includegraphics[scale=0.4, trim = 2cm 6cm 1cm
                            7.5cm,
                            clip]{./Bilder/Termin8/fehlerfrei_hochlaufen_Daubechies_Wavlet_lvl_9}
                            %FIXME [width=640px,
                             %height=474px]
                            \caption{Daubechies-Wavelet Zerlegung, Level 9}
                        \end{figure}
                    \vspace{-1.5em}
    
                    \end{minipage}
    
                \end{tabular}
                \end{center}
    
                %ende Vergleich Level 4
                
                
                    \begin{center}
                \begin{tabular}{ll}
    
                \hspace{-8em}
                    \begin{minipage}{0.6\textwidth}
    
                        \begin{figure}[H]
                            \label{fig:}
                            \includegraphics[scale=0.4, trim = 2cm 6cm 1cm
                            7.5cm,
                            clip]{./Bilder/Termin8/fehlerfrei_hochlaufen_Haar_Wavlet_lvl_12}
                            %FIXME [width=640px,
                             %height=474px]
                            \caption{Haar-Wavelet Zerlegung, Level 12}
                        \end{figure}
    
                    \end{minipage}
                    \begin{minipage}{0.6\textwidth}
    
                        \begin{figure}[H]
                            \label{fig:}
                            \includegraphics[scale=0.4, trim = 2cm 6cm 1cm
                            7.5cm,
                            clip]{./Bilder/Termin8/fehlerfrei_hochlaufen_Daubechies_Wavlet_lvl_12}
                            %FIXME [width=640px,
                             %height=474px]
                            \caption{Daubechies-Wavelet Zerlegung, Level 12}
                        \end{figure}
                    \vspace{-1.5em}
    
                    \end{minipage}
    
                \end{tabular}
                \end{center}
                
                %ende Vergleich fehlerfrei_hochlaufen_
                
                
                        \begin{center}
                \begin{tabular}{ll}
    
                \hspace{-8em}
                    \begin{minipage}{0.6\textwidth}
    
                        \begin{figure}[H]
                            \label{fig:}
                            \includegraphics[scale=0.4, trim = 2cm 6cm 1cm
                            7.5cm,
                            clip]{./Bilder/Termin8/fehlerfrei_hochlaufen_Spektrum.pdf}
                            %FIXME [width=640px,
                             %height=474px]
                            \caption{Strom-Spektrum bei hochlaufendem Motor}
                        \end{figure}
    
                    \end{minipage}
                    \begin{minipage}{0.6\textwidth}
    
                        \begin{figure}[H]
                            \label{fig:}
                            \includegraphics[scale=0.4, trim = 2cm 6cm 1cm
                            7.5cm,
                            clip]{./Bilder/Termin8/fehlerfrei_hochlaufen_Spectrogam.pdf}
                            %FIXME [width=640px,
                             %height=474px]
                            \caption{Strom-Spectrogram bei hochlaufendem Motor}
                        \end{figure}
                    \vspace{-1.5em}
    
                    \end{minipage}
    
                \end{tabular}
                \end{center}
                
                \vspace{2em}
                
                Sowohl in der Zerlegung durch ein Haar-, als auch in der
                Zerlegung mit einem Daubechies-Wavelet kann man sehen, dass die
                Approximationskurven, wie erwartet den Originalsignalverlauf
                annehmen, auch wenn Unterschiede in der Amplitude vorhanden
                sind. Das relevante an diesen Darstellungen ist erst ab Level 9
                sichtbar. Im 9. Level kann man erkennen, das die Detailkurven in
                beiden Zerlegungen einen Ausschlag in dem Zeitpunkt des
                Motoranschaltens vorweisen. Bei dem Haar-Wavelet ist dieser Peak
                negativ, bei dem Daubechies-Wavelet dagegen im 9. Level erst
                positiv, im 12. Level aber sowohl in den negativen als auch in
                den positiven Bereich ausgeprägt. Diese Peaks in den
                Detailkurven verdeutlichen damit den Anschaltvorgang des
                Universalmotors.\\
    
                
                Im Zeitsignal des Spektrums kann man zunächst den hohen
                Anlaufstrom des Motors erkennen. Die Bandberite im
                Amplitudenspektrum ist relativ klein und die kleinsten
                Frequenzen besitzen dementsprechend die größten Amplituden.\\
                Das Spektrogram, in dem nur nennenswerte Frequenzen bis zu ca.
                $1kHz$ zu erkennen sind, enthält auch andere Frequenzanteile
                (deutlich durch die gelbe bishin zur türkisen Farbe), welche
                wir als Rauschfrequenzen definieren. Dass am Anfang der
                Zeitachse die gelben Bereiche häufiger vorkommen, als am Ende, 
                wenn der Strom gesättigt ist, liegt daran,
                dass im Anlaufstrom große Veränderungen vorhanden sind, wodurch
                auch größere Frequenzanteile entstehen.
     
                %Ende Vergleich Level
                \vspace{1em}
                
                Als nächstes werden die Stromsignale im stationären Zustand
                eines fehlerfreien Motors und eines Motors mit Lamellenfehler
                anhand der beiden verwendeten Wavelets zerlegt und untersucht.
                Zuerst werden die Zerlegungen der fehlerfreien Motorstroms
                diskutiert:
                
                 \begin{center}
                \begin{tabular}{ll}
    
                \hspace{-8em}
                    \begin{minipage}{0.6\textwidth}
    
                        \begin{figure}[H]
                            \label{fig:}
                            \includegraphics[scale=0.4, trim = 2cm 6cm 1cm
                            7.5cm,
                            clip]{./Bilder/Termin8/fehlerfrei_gesaettigt_Haar_Wavlet_lvl_1}
                            %FIXME [width=640px,
                             %height=474px]
                            \caption{Haar-Wavelet Zerlegung, Level 1}
                        \end{figure}
    
                    \end{minipage}
                    \begin{minipage}{0.6\textwidth}
    
                        \begin{figure}[H]
                            \label{fig:}
                            \includegraphics[scale=0.4, trim = 2cm 6cm 1cm
                            7.5cm,
                            clip]{./Bilder/Termin8/fehlerfrei_gesaettigt_Daubechies_Wavlet_lvl_1}
                            %FIXME [width=640px,
                             %height=474px]
                            \caption{Daubechies-Wavelet Zerlegung, Level 1}
                        \end{figure}
                    \vspace{-1.5em}
    
                    \end{minipage}
    
                \end{tabular}
                \end{center}
                
                %Ende Vergleich Level 1
       
        \begin{center}
                \begin{tabular}{ll}
    
                \hspace{-8em}
                    \begin{minipage}{0.6\textwidth}
    
                        \begin{figure}[H]
                            \label{fig:}
                            \includegraphics[scale=0.4, trim = 2cm 6cm 1cm
                            7.5cm,
                            clip]{./Bilder/Termin8/fehlerfrei_gesaettigt_Haar_Wavlet_lvl_3}
                            %FIXME [width=640px,
                             %height=474px]
                            \caption{Haar-Wavelet Zerlegung, Level 3}
                        \end{figure}
    
                    \end{minipage}
                    \begin{minipage}{0.6\textwidth}
    
                        \begin{figure}[H]
                            \label{fig:}
                            \includegraphics[scale=0.4, trim = 2cm 6cm 1cm
                            7.5cm,
                            clip]{./Bilder/Termin8/fehlerfrei_gesaettigt_Daubechies_Wavlet_lvl_3}
                            %FIXME [width=640px,
                             %height=474px]
                            \caption{Daubechies-Wavelet Zerlegung, Level 3}
                        \end{figure}
                    \vspace{-1.5em}
    
                    \end{minipage}
    
                \end{tabular}
                \end{center}
                
                %Ende Vergleich Level 2
                
                
          \begin{center}
                \begin{tabular}{ll}
    
                \hspace{-8em}
                    \begin{minipage}{0.6\textwidth}
    
                        \begin{figure}[H]
                            \label{fig:}
                            \includegraphics[scale=0.4, trim = 2cm 6cm 1cm
                            7.5cm,
                            clip]{./Bilder/Termin8/fehlerfrei_gesaettigt_Haar_Wavlet_lvl_6}
                            %FIXME [width=640px,
                             %height=474px]
                            \caption{Haar-Wavelet Zerlegung, Level 6}
                        \end{figure}
    
                    \end{minipage}
                    \begin{minipage}{0.6\textwidth}
    
                        \begin{figure}[H]
                            \label{fig:}
                            \includegraphics[scale=0.4, trim = 2cm 6cm 1cm
                            7.5cm,
                            clip]{./Bilder/Termin8/fehlerfrei_gesaettigt_Daubechies_Wavlet_lvl_6}
                            %FIXME [width=640px,
                             %height=474px]
                            \caption{Daubechies-Wavelet Zerlegung, Level 6}
                        \end{figure}
                    \vspace{-1.5em}
    
                    \end{minipage}
    
                \end{tabular}
                \end{center}
    
                %ende Vergleich Level 3
                
                
        \begin{center}
                \begin{tabular}{ll}
    
                \hspace{-8em}
                    \begin{minipage}{0.6\textwidth}
    
                        \begin{figure}[H]
                            \label{fig:}
                            \includegraphics[scale=0.4, trim = 2cm 6cm 1cm
                            7.5cm,
                            clip]{./Bilder/Termin8/fehlerfrei_gesaettigt_Haar_Wavlet_lvl_9}
                            %FIXME [width=640px,
                             %height=474px]
                            \caption{Haar-Wavelet Zerlegung, Level 9}
                        \end{figure}
    
                    \end{minipage}
                    \begin{minipage}{0.6\textwidth}
    
                        \begin{figure}[H]
                            \label{fig:}
                            \includegraphics[scale=0.4, trim = 2cm 6cm 1cm
                            7.5cm,
                            clip]{./Bilder/Termin8/fehlerfrei_gesaettigt_Daubechies_Wavlet_lvl_9}
                            %FIXME [width=640px,
                             %height=474px]
                            \caption{Daubechies-Wavelet Zerlegung, Level 9}
                        \end{figure}
                    \vspace{-1.5em}
    
                    \end{minipage}
    
                \end{tabular}
                \end{center}
    
                %ende Vergleich Level 4
                
                
                    \begin{center}
                \begin{tabular}{ll}
    
                \hspace{-8em}
                    \begin{minipage}{0.6\textwidth}
    
                        \begin{figure}[H]
                            \label{fig:}
                            \includegraphics[scale=0.4, trim = 2cm 6cm 1cm
                            7.5cm,
                            clip]{./Bilder/Termin8/fehlerfrei_gesaettigt_Haar_Wavlet_lvl_12}
                            %FIXME [width=640px,
                             %height=474px]
                            \caption{Haar-Wavelet Zerlegung, Level 12}
                        \end{figure}
    
                    \end{minipage}
                    \begin{minipage}{0.6\textwidth}
    
                        \begin{figure}[H]
                            \label{fig:}
                            \includegraphics[scale=0.4, trim = 2cm 6cm 1cm
                            7.5cm,
                            clip]{./Bilder/Termin8/fehlerfrei_gesaettigt_Daubechies_Wavlet_lvl_12}
                            %FIXME [width=640px,
                             %height=474px]
                            \caption{Daubechies-Wavelet Zerlegung, Level 12}
                        \end{figure}
                    \vspace{-1.5em}
    
                    \end{minipage}
    
                \end{tabular}
                \end{center}
                
                
                                        \begin{center}
                \begin{tabular}{ll}
    
                \hspace{-8em}
                    \begin{minipage}{0.6\textwidth}
    
                        \begin{figure}[H]
                            \label{fig:}
                            \includegraphics[scale=0.4, trim = 2cm 6cm 1cm
                            7.5cm,
                            clip]{./Bilder/Termin8/fehlerfrei_gesaettig_Spektrum.pdf}
                            %FIXME [width=640px,
                             %height=474px]
                            \caption{Strom-Spektrum, fehlerfrei und gesättigt}
                        \end{figure}
    
                    \end{minipage}
                    \begin{minipage}{0.6\textwidth}
    
                        \begin{figure}[H]
                            \label{fig:}
                            \includegraphics[scale=0.4, trim = 2cm 6cm 1cm
                            7.5cm,
                            clip]{./Bilder/Termin8/fehlerfrei_gesaettig_Spectrogam.pdf}
                            %FIXME [width=640px,
                             %height=474px]
                            \caption{Strom-Spectrogram, fehlerfrei und
                            gesättigt}
                        \end{figure}
                    \vspace{-1.5em}
    
                    \end{minipage}
    
                \end{tabular}
                \end{center}
                
                %ende Vergleich fehlerfrei_gesaettigt_
                \vspace{2em}
                
                An den Zerlegungen des Stromsignals anhand beiden Wavelets kann
                man zunächst erkennen, dass die Approximationen stets das
                komplette Originalsignal überlagern, da sie den selben Verlauf
                mit einer höheren Amplitude annehmen. Die Details haben bei
                beiden Wavelets eine relativ kleine Amplitude bis zum 3. Level,
                wobei die Amplituden der Daubechies-Wavelets-Details stets
                eitwas niedriger ausfallen. Ab dem 6.Level, steigen die
                Amplituden an, bis sie im 9. bis 12.Level fast so hoch sind, wie
                die Amplituden der Approximationen. Auch der Verlauf der Details
                ähnelt sich stark dem Stromverlauf im gesättigten Zustand.\\
                
                Das Spektrum zeigt uns ein erwartetes Strom-Zeitsignal, wobei
                das Amplitudenspektrum Frequenzanteile bis zu $4kHz$ aufweist.
                Das Spektrogramm zeigt Frequenzanteile bis zu $100Hz$, wobei
                auch hier ein leichter Übergang von rot zu gelb bis zu ca.
                $300Hz$ vorhanden ist. Den blauen Bereich des Spektrogramms kann
                man erneut als Rauschen definieren.
                
                \vspace{2em}    
                            
                Als nächtes folgen die Zerlegungen des Stroms bei dem mit
                Lamellenfehler belasteten Motor:
                
                 \begin{center}
                \begin{tabular}{ll}
    
                \hspace{-8em}
                    \begin{minipage}{0.6\textwidth}
    
                        \begin{figure}[H]
                            \label{fig:}
                            \includegraphics[scale=0.4, trim = 2cm 6cm 1cm
                            7.5cm,
                            clip]{./Bilder/Termin8/lamellenfehler_gesaettigt_Haar_Wavlet_lvl_1}
                            %FIXME [width=640px,
                             %height=474px]
                            \caption{Haar-Wavelet Zerlegung, Level 1}
                        \end{figure}
    
                    \end{minipage}
                    \begin{minipage}{0.6\textwidth}
    
                        \begin{figure}[H]
                            \label{fig:}
                            \includegraphics[scale=0.4, trim = 2cm 6cm 1cm
                            7.5cm,
                            clip]{./Bilder/Termin8/lamellenfehler_gesaettigt_Daubechies_Wavlet_lvl_1}
                            %FIXME [width=640px,
                             %height=474px]
                            \caption{Daubechies-Wavelet Zerlegung, Level 1}
                        \end{figure}
                    \vspace{-1.5em}
    
                    \end{minipage}
    
                \end{tabular}
                \end{center}
                
                %Ende Vergleich Level 1
       
        \begin{center}
                \begin{tabular}{ll}
    
                \hspace{-8em}
                    \begin{minipage}{0.6\textwidth}
    
                        \begin{figure}[H]
                            \label{fig:}
                            \includegraphics[scale=0.4, trim = 2cm 6cm 1cm
                            7.5cm,
                            clip]{./Bilder/Termin8/lamellenfehler_gesaettigt_Haar_Wavlet_lvl_3}
                            %FIXME [width=640px,
                             %height=474px]
                            \caption{Haar-Wavelet Zerlegung, Level 3}
                        \end{figure}
    
                    \end{minipage}
                    \begin{minipage}{0.6\textwidth}
    
                        \begin{figure}[H]
                            \label{fig:}
                            \includegraphics[scale=0.4, trim = 2cm 6cm 1cm
                            7.5cm,
                            clip]{./Bilder/Termin8/lamellenfehler_gesaettigt_Daubechies_Wavlet_lvl_3}
                            %FIXME [width=640px,
                             %height=474px]
                            \caption{Daubechies-Wavelet Zerlegung, Level 3}
                        \end{figure}
                    \vspace{-1.5em}
    
                    \end{minipage}
    
                \end{tabular}
                \end{center}
                
                %Ende Vergleich Level 2
                
                
          \begin{center}
                \begin{tabular}{ll}
    
                \hspace{-8em}
                    \begin{minipage}{0.6\textwidth}
    
                        \begin{figure}[H]
                            \label{fig:}
                            \includegraphics[scale=0.4, trim = 2cm 6cm 1cm
                            7.5cm,
                            clip]{./Bilder/Termin8/lamellenfehler_gesaettigt_Haar_Wavlet_lvl_6}
                            %FIXME [width=640px,
                             %height=474px]
                            \caption{Haar-Wavelet Zerlegung, Level 6}
                        \end{figure}
    
                    \end{minipage}
                    \begin{minipage}{0.6\textwidth}
    
                        \begin{figure}[H]
                            \label{fig:}
                            \includegraphics[scale=0.4, trim = 2cm 6cm 1cm
                            7.5cm,
                            clip]{./Bilder/Termin8/lamellenfehler_gesaettigt_Daubechies_Wavlet_lvl_6}
                            %FIXME [width=640px,
                             %height=474px]
                            \caption{Daubechies-Wavelet Zerlegung, Level 6}
                        \end{figure}
                    \vspace{-1.5em}
    
                    \end{minipage}
    
                \end{tabular}
                \end{center}
    
                %ende Vergleich Level 3
                
                
        \begin{center}
                \begin{tabular}{ll}
    
                \hspace{-8em}
                    \begin{minipage}{0.6\textwidth}
    
                        \begin{figure}[H]
                            \label{fig:}
                            \includegraphics[scale=0.4, trim = 2cm 6cm 1cm
                            7.5cm,
                            clip]{./Bilder/Termin8/lamellenfehler_gesaettigt_Haar_Wavlet_lvl_9}
                            %FIXME [width=640px,
                             %height=474px]
                            \caption{Haar-Wavelet Zerlegung, Level 9}
                        \end{figure}
    
                    \end{minipage}
                    \begin{minipage}{0.6\textwidth}
    
                        \begin{figure}[H]
                            \label{fig:}
                            \includegraphics[scale=0.4, trim = 2cm 6cm 1cm
                            7.5cm,
                            clip]{./Bilder/Termin8/lamellenfehler_gesaettigt_Daubechies_Wavlet_lvl_9}
                            %FIXME [width=640px,
                             %height=474px]
                            \caption{Daubechies-Wavelet Zerlegung, Level 9}
                        \end{figure}
                    \vspace{-1.5em}
    
                    \end{minipage}
    
                \end{tabular}
                \end{center}
    
                %ende Vergleich Level 4
                
                
                    \begin{center}
                \begin{tabular}{ll}
    
                \hspace{-8em}
                    \begin{minipage}{0.6\textwidth}
    
                        \begin{figure}[H]
                            \label{fig:}
                            \includegraphics[scale=0.4, trim = 2cm 6cm 1cm
                            7.5cm,
                            clip]{./Bilder/Termin8/lamellenfehler_gesaettigt_Haar_Wavlet_lvl_12}
                            %FIXME [width=640px,
                             %height=474px]
                            \caption{Haar-Wavelet Zerlegung, Level 12}
                        \end{figure}
    
                    \end{minipage}
                    \begin{minipage}{0.6\textwidth}
    
                        \begin{figure}[H]
                            \label{fig:}
                            \includegraphics[scale=0.4, trim = 2cm 6cm 1cm
                            7.5cm,
                            clip]{./Bilder/Termin8/lamellenfehler_gesaettigt_Daubechies_Wavlet_lvl_12}
                            %FIXME [width=640px,
                             %height=474px]
                            \caption{Daubechies-Wavelet Zerlegung, Level 12}
                        \end{figure}
                    \vspace{-1.5em}
    
                    \end{minipage}
    
                \end{tabular}
                \end{center}
                
                \begin{center}
                \begin{tabular}{ll}
    
                \hspace{-8em}
                    \begin{minipage}{0.6\textwidth}
    
                        \begin{figure}[H]
                            \label{fig:}
                            \includegraphics[scale=0.4, trim = 2cm 6cm 1cm
                            7.5cm,
                            clip]{./Bilder/Termin8/lamellenfehler_gesaettig_Spektrum.pdf}
                            %FIXME [width=640px,
                             %height=474px]
                            \caption{Strom-Spektrum, mit Lamellenfehler und
                            gesättigt}
                        \end{figure}
    
                    \end{minipage}
                    \begin{minipage}{0.6\textwidth}
    
                        \begin{figure}[H]
                            \label{fig:}
                            \includegraphics[scale=0.4, trim = 2cm 6cm 1cm
                            7.5cm,
                            clip]{./Bilder/Termin8/lamellenfehler_gesaettig_Spectrogam.pdf}
                            %FIXME [width=640px,
                             %height=474px]
                            \caption{Strom-Spectrogram, mit Lamellenfehler und
                            gesättigt}
                        \end{figure}
                    \vspace{-1.5em}
    
                    \end{minipage}
    
                \end{tabular}
                \end{center}
                
                %ende Vergleich Level 5 lamellenfehler_gesaettigt_
                
                Auch bei den Zelegungen des Stromsignals des Motors mit
                Lamellenfehler kann man die Originalsignale nicht mehr sehen, da
                sie von den Approximationskurven mit gleichem Verlauf und
                höheren Amplituden überlagert werden. Die Detailkurven werden bis zu Level 9
                konstant in der Amplitude größer. In Level 12 kann man dann
                auffällige Welligkeiten im Verlauf feststellen. Abhängig von der
                Zeit entstehen Wellenbäuche und -täler, welche auf den
                Lamellenfehler im Motor zurückzuführen sind. Da diese Art
                von Auffälligkeiten in der Zerlegung von gesättigten
                Stromsignalen auftreten, ist der Einsatz von Waveltes in der
                Fehlerdiagnose von Motoren möglich.\\
                
                Ähnliche Welligkeiten machen sich auch im Amplitudenspektrum
                bemerkbar. Im Spektrogramm dagegen ist keine Auffälligkeit
                wahrnehmbar, außer dass der Farbverlauf von rot (den
                nennenswerten Frequenzen) zu gelb (weniger relevante Frequenzen)
                ein wenig unklarer Verläuft als beim Spektrogramm des
                fehlerfreien Motors.
                
                
    \end{quote}  % Ende Subsection Auswertung Termin 8
\end{quote} %Ende section

%--------------------------------------------------------------------
%--------------------------------------------------------------------    


%--------------------------------------------------------------------
%--------------------------------------------------------------------
\section{Quellcodes}
\begin{quote}

    \subsection{Codes aus Termin 7}
    \begin{quote}
        
        \subsubsection{Frequenzverlauf über der Zeit}
        \begin{quote}
            \lstinputlisting[
                caption={frequenz imZeitbereich ausSignal},
                label=lst:Matlab]
                {./Matlab/Termin7/frequenz_imZeitbereich_ausSignal.m}
        \end{quote}

        \subsubsection{Frequenzverlauf aus dem Spektrogram}
        \begin{quote}
            \lstinputlisting[
                caption={frequenz durch Spektogramm},
                label=lst:Matlab]
                {./Matlab/Termin7/frequenz_durch_Spektogramm.m}
        \end{quote}

        \subsubsection{Algorithmus zur Drehfrequenzberechnung}
        \begin{quote}
            \lstinputlisting[
                caption={Algorithmus zur Drehfrequenzberechnung},
                label=lst:Matlab]
                {./Matlab/Termin7/Vorbereitungsaufgabe4.m}
        \end{quote}
        
    \end{quote}%ende Codes aus Termin7
    
    \subsection{Codes aus Termin 8}
    \begin{quote}
    
       \subsubsection{Funktion haardec.m}
       \begin{quote}
       \lstinputlisting[
                caption={Funktion haardec.m},
                label=lst:Matlab]
                {./Matlab/Termin8/haardec_8_1_1.m}
       \end{quote}
       
       \subsubsection{Funktion haardeclevel.m}
       \begin{quote}
       \lstinputlisting[
                caption={Funktion haardeclevel.m},
                label=lst:Matlab]
                {./Matlab/Termin8/haardeclevel_8_1_2.m}
       \end{quote}
       
       \subsubsection{Funktion getAppDet.m}
       \begin{quote}
       \lstinputlisting[
                caption={Funktion getAppDet.m},
                label=lst:Matlab]
                {./Matlab/Termin8/getAppDet_8_1_3.m}
       \end{quote}
       
       \subsubsection{Funktion Daubechies\_Wavelets.m}
       \begin{quote}
       \lstinputlisting[
                caption={Funktion Daubechies\_Wavelets.m},
                label=lst:Matlab]
                {./Matlab/Termin8/Daubechies_Wavelets.m}
       \end{quote}
       
       
       \subsubsection{Vorbereitungsaufgaben Plots}
       \begin{quote}
       \lstinputlisting[
                caption={Aufrufe für die Plots},
                label=lst:Matlab]
                {./Matlab/Termin8/Vorbereitungsaufgaben_8.m}
       \end{quote}
\end{quote}

%--------------------------------------------------------------------
%--------------------------------------------------------------------


%\begin{thebibliography}{999}
%\bibitem {Schaltungwandlerbox} Prof. Dr.-Ing. Gühmann, Clemens; Dipl.-Ing. Funk, Jürgen: MDVLaborGeraete_web, S.4

%Name, Vorname.; evtl. Name2, Vorname2.: Titel des Dokumentes
%oder Buches, Zeitschrift/Verlag/URL (Auflage, Erscheinungsort, -jahr), ggf. Seitenzahlen
%\bibitem {PasevalscheTheorem} \url{https://de.wikipedia.org/wiki/Parsevalsches_Theorem}, Zugriff
%23.05.2012
%\end{thebibliography}


\end{document}


