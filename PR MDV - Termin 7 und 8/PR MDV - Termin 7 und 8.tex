\newcommand{\institut}{Institut f\"ur Energie und  Automatisiertungstechnik}
\newcommand{\fachgebiet}{Elektronische Mess- und Diagnosetechnik}
\newcommand{\veranstaltung}{Praktikum Messdatenverarbeitung}
\newcommand{\pdfautor}{\"Ozg\"u Dogan (326 048), Timo Lausen (325 411), Boris Henckell (325 779)}
\newcommand{\autor}{\"Ozg\"u Dogan (326 048)\\ Timo Lausen (325 411)\\ Boris Henckell (325 779)}
\newcommand{\pdftitle}{Praktikum Messdatenverarbeitung  Termin 7}
\newcommand{\prototitle}{Praktikum Messdatenverarbeitung \\ Termin 7}
\newcommand{\aufgabe}{}

\newcommand{\gruppe}{Gruppe: G1 Fr 08-10}
\newcommand{\betreuer}{Betreuer: J\"urgen Funk}

\input{../../packages/tu_header_8}
%\begin{document}

% \lstlistoflistings
\definecolor{darkgray}{rgb}{0.95,0.95,0.95}
\definecolor{darkolivegreen}{HTML}{01a801}
\definecolor{functionsBlue}{HTML}{32b9b9}
\definecolor{variableRed}{rgb}{1,0,0}
\definecolor{stringBrown}{HTML}{bc8e8e} % f geht nicht

\lstset{
        %\lstset{extendedchars=true} % Umlaute an der richtigen stelle und nicht am Anfang ausgeben
        %basicstyle=\footnotesize\ttfamily,
        basicstyle=\small,
        %
        inputencoding=utf8,
        %
        tabsize=4,
        showspaces=false,
        showtabs=false,
        showstringspaces=true, % no special string spaces
        %
        backgroundcolor=\color{darkgray}, % background
        stringstyle=\color{stringBrown}\fseries, % Strings
        keywordstyle=\color{functionsBlue}\bfseries, % keywords Blau
        identifierstyle=\color{variableRed}, % variablen
        commentstyle=\color{darkolivegreen}, %  comments
        %
        breaklines=true,
        %
        numbers=left,
        numberstyle=\tiny,
        stepnumber=1,
        numbersep=7pt,
        %
        frame=single,
        columns=flexible,
        %
        xleftmargin=-2cm,
        xrightmargin=-1.5cm,
        %
        language=Matlab
}


%---------------------------------------------------------------------
%---------------------------------------------------------------------
%---------------------------------------------------------------------


\section{Vorbereitungsaufgaben}
\begin{quote}
    
    \subsection{Vorbereitungsaufgaben zu Termin 7}
    \begin{quote}
 
	
	\end{quote}%Ende Vorbereitungsaufgaben Termin 7
	
	
   %\subsection{Vorbereitungsaufgaben zu Termin 8}
    %\begin{quote}
 
	%\end{quote}%Ende Vorbereitungsaufgaben zu Termin 8	
\end{quote}%Ende Vorbereitungsaufgaben

% \section{Durchführungen}
% \begin{quote}
% 		
% 		\subsection{Durchführung zu Termin 7}
% 		\begin{quote}
% 			
% 		\end{quote}%Ende der Durchführung von Termin 7
% 		
% 		\subsection{Durchführung zu Termin 8}
% 		\begin{quote}
% 		
% 		\end{quote}%Ende der Durchführungen von Termin 8
% 		
% \end{quote}%Ende Durchführungen


%--------------------------------------------------------------------
%--------------------------------------------------------------------
% \section{Auswertung}
% \begin{quote}
% 	\subsection{Auswertung Termin 7}
%     \begin{quote}
% 
%     \end{quote}  % Ende Subsection Auswertung Termin 7
%     
%     \subsection{Auswertung Termin 8}
%     \begin{quote}
%     
%    
%         
%     \end{quote}  % Ende Subsection Auswertung Termin 8
% \end{quote} %Ende section

%--------------------------------------------------------------------
%--------------------------------------------------------------------    


%--------------------------------------------------------------------
%--------------------------------------------------------------------
\section{Quellcodes}
\begin{quote}

	\subsection{Codes aus Termin 7}
	\begin{quote}
% 	    \subsubsection{FIRfilterung.m}
% 	    \begin{quote}
% 	        \lstinputlisting[
% 	            caption={FIRfilterung},
% 	            label=lst:Matlab]
% 	            {./Matlab/FIRfilterung.m}
% 	    \end{quote}
% 	    
% 	    \subsubsection{getFIRTiefpass.m}
% 	    \begin{quote}
% 	        \lstinputlisting[
% 	            caption={getFIRTiefpass},
% 	            label=lst:Matlab]
% 	            {./Matlab/getFIRTiefpass.m}
% 	    \end{quote}
% 
%         \subsubsection{Testfunktion.m}
%         \begin{quote}
%             \lstinputlisting[
%                 caption={Testfunktion},
%                 label=lst:Matlab]
%                 {./Matlab/Testfunktion.m}
%         \end{quote}
% 	    
% 	    \subsubsection{DecimFilt.m}
% 	    \begin{quote}
% 	        \lstinputlisting[
% 	            caption={DecimFilt},
% 	            label=lst:Matlab]
% 	            {./Matlab/DecimFilt.m}
% 	    \end{quote}
	    
	\end{quote}
\end{quote}

%--------------------------------------------------------------------
%--------------------------------------------------------------------


%\begin{thebibliography}{999}
%\bibitem {Schaltungwandlerbox} Prof. Dr.-Ing. Gühmann, Clemens; Dipl.-Ing. Funk, Jürgen: MDVLaborGeraete_web, S.4

%Name, Vorname.; evtl. Name2, Vorname2.: Titel des Dokumentes
%oder Buches, Zeitschrift/Verlag/URL (Auflage, Erscheinungsort, -jahr), ggf. Seitenzahlen
%\bibitem {PasevalscheTheorem} \url{https://de.wikipedia.org/wiki/Parsevalsches_Theorem}, Zugriff
%23.05.2012
%\end{thebibliography}


\end{document}


